\documentclass{article}

\usepackage{indentfirst}

\title{\textbf{Analyse de données d’eye-tracking en Réalité Virtuelle}}
\author{\Large{Adonis Stavridis}}
\date{Février 2021}

\renewcommand{\contentsname}{Sommaire}

\begin{document}

  % ----------------------------------------------------------------------------
  % TITLEPAGE
  % ----------------------------------------------------------------------------
  
  \maketitle
  \tableofcontents
  \pagebreak

  % ----------------------------------------------------------------------------
  % INTRODUCTION
  % ----------------------------------------------------------------------------

  \section{Introduction}
  Le eye-tracking est une technologie assez récente qui étudie les mouvements
  oculaires d'un individu. Elle établie une nouvelle interface entre Homme et
  machine et ouvre la porte à des utilisations dans plusieurs domaines dont 
  l'analyse de documents visuels ou graphiques, le divertissement, les jeux
  vidéos. Nous nous intéresserons ici surtout sur la réalité virtuelle car les
  recherches de la technologie d'eye-tracking dans ce domaine-ci sont peu
  nombreuses.

  \bigskip
  Cette technologie permet de déterminer la position du regard d'un individu sur
  un écran. Cela permet d'effectuer des études quantitatives et qualitatives sur
  un document graphique par exemple, et ainsi énumérer ses élements attrayants 
  ou non, afin de l'améliorer. L'eye-tracking s'avère donc être très pratique 
  dans le domaine de la publicité et du design. Cependant, certaines études ont
  besoin d'un environment plus réalistes. La réalité virtuelle permet alors de
  virtualiser tout environment et d'ajouter une nouvelle couche d'immersion dans
  un monde virtuel. Effectuer des études d'eye-tracking dans un environment plus
  proche de la réalité peut donner des résultats beaucoup plus fiables.

  \bigskip
  L'utilisation d'une technologie d'eye-tracking dans un environment virtuel
  permet une multitude de possibilités, auparavant impossibles, pour étudier et 
  comprendre le comportement humain lorsqu'il est placé dans différentes
  situations.

  % ----------------------------------------------------------------------------
  % MATERIEL
  % ----------------------------------------------------------------------------

  \section{Matériel}

  \subsection{Tobii}

  \subsection{HTC Vive}

  % ----------------------------------------------------------------------------
  % LOGICIELS
  % ----------------------------------------------------------------------------

  \section{Logiciels}

  \subsection{PyGaze}

  \subsection{GazeParser}

  % ----------------------------------------------------------------------------
  % CONCLUSION
  % ----------------------------------------------------------------------------

  \section{Conclusion}

\end{document}
