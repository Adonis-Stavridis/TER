\documentclass{article}

\usepackage{lmodern}
\usepackage[utf8]{inputenc}
\usepackage[T1]{fontenc}
\usepackage[french]{babel}
\usepackage{url}

\title{\textbf{Analyse de données d’eye-tracking en Réalité Virtuelle}}
\author{\Large{Adonis Stavridis}}
\date{Février 2021}

\begin{document}

% ----------------------------------------------------------------------------
% TITLEPAGE
% ----------------------------------------------------------------------------

\maketitle
\tableofcontents
\pagebreak

% ----------------------------------------------------------------------------
% INTRODUCTION
% ----------------------------------------------------------------------------

\section{Introduction}
L'eye-tracking \cite{wiki:eye_tracking}, ou oculométrie, est une technologie
assez récente qui détermine la position du regard d'un individu. Des capteurs
spéciaux envoient des rayons infrarouges vers les yeux d'un individu et ceux-ci
sont réfléchis, leur permettant ainsi de déterminer ses mouvements oculaires
sur un écran. Elle établie alors une nouvelle interface entre Homme et machine
et est devenue aujourd'hui une technologie principale dans des études liées au
système visuel humain, à la psychologie, au marketing et design. Elle est en
fait dèjà très utilisée dans les jeux vidéos. Un domaine où cette technologie
reste encore peu developpée est la réalité virtuelle.

\bigskip
Déterminer la position du regard d'un individu sur un écran permet d'effectuer des études quantitatives et qualitatives sur de multiples supports, et ainsi
comprendre les comportements humains dans différentes situations. L'eye-tracking
s'avère donc être très pratique pour étudier sur un document par exemple, les 
zones qui sont le plus attrayantes et celles qui le sont moins. Cependant, 
certaines études ont besoin d'un environment plus réalistes. La réalité 
virtuelle ajoute une nouvelle couche d'immersion, permettant à un individu de
se sentir et agir de façon plus réaliste. Ainsi, la réalité virtuelle 
permettrait de livrer des résultats beaucoup plus fiables pour certains 
domaines, et ainsi aider à l'avancement des recherches sur le comportement
humain.

% ----------------------------------------------------------------------------
% MATERIEL
% ----------------------------------------------------------------------------

\section{Capteurs d'oculométrie}

Tout d'abord, pour effectuer des mesures d'oculométrie, il faudrait énumérer les
différents produits et capteurs d'eye-tracking disponibles et les comparer, afin de déterminer lesquels sont les plus intéressants ou ambitieux.

\subsection{Tobii}

\cite{tobii}
\cite{yt:tobii_vr}

\subsection{HTC Vive}

% ----------------------------------------------------------------------------
% LOGICIELS
% ----------------------------------------------------------------------------

\section{Logiciels}

\subsection{PyGaze}

\subsection{GazeParser}

% ----------------------------------------------------------------------------
% CONCLUSION
% ----------------------------------------------------------------------------

\section{Conclusion}

% ----------------------------------------------------------------------------
% BIBLIOGRAPHIE
% ----------------------------------------------------------------------------

\pagebreak
\bibliographystyle{unsrt}
\bibliography{recherches}

\end{document}
