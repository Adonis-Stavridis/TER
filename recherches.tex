\documentclass[12pt]{article}

\usepackage{lmodern}
\usepackage[utf8]{inputenc}
\usepackage[T1]{fontenc}
\usepackage[french]{babel}
\usepackage{url}
\usepackage{graphicx}
\graphicspath{ {./img/} }

\title{\textbf{Analyse de données d’eye-tracking en Réalité Virtuelle}}
\author{\Large{Adonis Stavridis}}
\date{Février 2021}

\begin{document}

% ----------------------------------------------------------------------------
% TITLEPAGE
% ----------------------------------------------------------------------------

\maketitle
\tableofcontents
\pagebreak

% ----------------------------------------------------------------------------
% INTRODUCTION
% ----------------------------------------------------------------------------

\section{Introduction}
L'eye-tracking \cite{wiki:eye_tracking}, ou oculométrie, est une technologie
assez récente qui détermine la position du regard d'un individu dans un
environment virtuel. Elle établie alors une nouvelle interface entre Homme et
machine et c'est devenue aujourd'hui une technologie principale dans des études
liées au système visuel humain, à la psychologie, au marketing et au design.
Elle est aussi dèjà très utilisée dans les jeux vidéos. L'eye-tracking est très
ambitieux, cependant les recherches dans le domaine de la réalité virtuelle
reste encore moindres. C'est donc dans ce domaine que l'on se concentrera
davantage.

\bigskip
Déterminer la position du regard d'un individu sur un écran permet d'effectuer
des études quantitatives et qualitatives sur de multiples supports, et ainsi
comprendre les comportements humains dans différentes situations. L'eye-tracking
s'avère donc être très pratique pour étudier sur un document par exemple, les
zones qui sont le plus attrayantes et celles qui le sont moins. Cependant,
certaines études ont besoin d'un environment plus réalistes. La réalité
virtuelle ajoute une nouvelle couche d'immersion, permettant à un individu de
se sentir et agir de façon plus réaliste. Ainsi, la réalité virtuelle
permettrait de livrer des résultats beaucoup plus fiables pour certains
domaines, et ainsi pousser à l'avancement des recherches sur le comportement
humain.

% ----------------------------------------------------------------------------
% CAPTEURS
% ----------------------------------------------------------------------------

\section{Capteurs d'oculométrie}

L'eye-tracking est permis grace à des capteurs spéciaux qui envoient des rayons
infrarouges vers les yeux d'un individu. Ils sont infrarouges car ils ne sont
pas visibles par la vision humain. Ces rayons sont alors réfléchis par les yeux
et les capteurs peuvent ainsi déterminer la direction de regard au fil du temps.
Ensuite un logiciel va déterminer la position du regard dans un environment, que
ce soit un écran ou un monde virtuel.

\bigskip
Il existe trois types de capteurs d'oculométrie sur la marché actuellement.

\begin{itemize}
  \item Des capteurs pour écrans : ce sont les capteurs les plus basiques. Ils
        sont placés au-dessus ou en-dessous d'un écran et orientés vers les yeux
        d'un individu. Ils permettent de suivre le regard sur un écran, donc
        dans un environment à deux dimensions.
  \item Des capteurs intégrés à des lunettes : ces capteurs suivent le regard
        d'un individu effectuant des taches dans le monde réel. Les capteurs
        sont alors orientés vers les yeux de l'individu et des caméras capturent
        ce que l'individu observe.
  \item Des capteurs intégrés à des casques de réalité virtuelle : ce sont des
        capteur similaires à ceux des écrans, mais qui ont été manipulés pour
        s'intégrer dans des casques de réalité virtuelle et observer les yeux à
        un distance beaucoup plus courte. Ces capteurs permettraient alors de
        suivre le regard dans un environment à trois dimensions. Un avantage à
        cette technologie ajoutée à la réalité virtuelle est le rendu fovéal
        \cite{wiki:foveated_rendering} : les yeux ne se focalisent qu'en une
        seule région tandis que le reste du champ visuel est flou. Cette méthode
        de rendu propose d'effectuer un rendu de très haute qualité à la région
        de focalisation mais une qualité de rendu moins important dans les reste
        du champ périphérique.
\end{itemize}

\bigskip
La plupart des produits liés à l'eye-tracking sont distribués par des
entreprises telles que Tobii \cite{tobii} et HTC Vice \cite{htc_vive_pro_eye}.
Ces solutions peuvent être utilisés dans plusieurs domaines : les capteurs pour
écrans peuvent être utile pour étudier le regard d'un individu sur une page web,
par exemple, et ainsi déterminer les régions les plus attrayantes ou non. Les
lunettes avec capteurs intégrées sont plutôt utilisés dans le monde
professionnel, dans les métiers de l'artisanal, dans la médicine ou le pilotage.
Les casques de réalité virtuelle avec capteurs intégrés seraient utilisés dans
des situations d'entrainement ou des circonstances pas facilement réalisables
dans le monde réel. Certains produits similaires peuvent aussi être utilisés
pour assister à des personnes en situatio de handicap et à mobilité réduite.

\bigskip
Il existe alors une grande variété des capteurs, certains plus spécialisés dans
des domaines que d'autres, mais ayants le même objectif : suivre le regard. Ce
suivi est d'autant plus utile pour effectuer des recherches. Les capteurs sont
fournis avec des logiciels capables de produire toutes les données nécessaires
pour les analyser.

% ----------------------------------------------------------------------------
% LOGICIELS
% ----------------------------------------------------------------------------

\section{Logiciels}

De nos jours, la plupart des gens possédent un appareil électronique avec une
caméra intergrée, que ce soit une webcam ou la caméra d'un smartphone. Il n'est
pas nécessaire d'acheter du matériel professionnel pour profiter de la
technologie d'eye-tracking. Il existe plusieurs logiciels et bibliothèques qui
permettent d'effectuer de l'oculométrie, facilement et gratuitement aussi.

\subsection{OpenCV}

L'une des bibliothèques les plus utilisées dans le traitement d'images en temps
réel et en vision par ordinateur ou Computer Vision est OpenCV \cite{opencv}.
Elle propose des fonctionnalités pour traiter des données brutes d'images, mais
aussi une grande variété d'outils issus de l'état de l'art en vision des
ordinateurs (détection de visage par exemple) et des algorithmes d'apprentissage
artificiel. En tout, c'est une bibliothèque très compléte, pour la majorité des
domaines de la vision par ordinateur, mais pas forcément spécialisée dans
l'oculométrie. Il est possible toutefois d'écrire des algorithmes pour suivre
le regard grace aux outils mis à disposition par la bibliothèque. Il existe
également de jeux de données assez grands mis à disposition en ligne, tel que
NVGaze \cite{nvgaze} de NVidia, pour entrainer des réseaux de neurones à
éstimer le regard grace à des images \cite{gaze_tracking}. OpenCV propose plein
d'outils de visualisation des données en temps réel, mais aucun pour leur
exploitation.

\bigskip
Le grand avantage de OpenCV est son age et popularité. Elle est la bibliothèque
de référence dans le domaine de la vision par ordinateur, propose parmi des
outils très optimisés, tout en restant bien documenté. C'est aussi un projet
open-source, et c'est pour cela qu'elle rassemble une grande communauté.
Puisqu'elle n'est pas spécialisée dans l'eye-tracking, elle ne propose pas
d'outils d'analyse et de visualisation de données, nécessaires en oculométrie.

\subsection{PyGaze}

Une autre bibliothèque très intéressante est PyGaze \cite{pygaze}. C'est une
toolbox permettant de suivre le regard sur un écran grace à une caméra ou
webcam. Elle propose également des outils pour analyser les données
enregistrées, lors d'une séance d'eye-tracking. Ces outils permettent de créer
plusieurs types d'images pour suivre le regard au fur et à mesure du temps.
Parmi ces images se trouvent des cartes de fixations, des heatmap et des schémas
du chemin du regard. Elle permettent de facilement comprendre les données et
d'en tirer des conclusions plus rapidement. En plus, la bibliothèque PyGaze peut
être utilisée avec la SDK de Tobii pour profiter de toutes les fonctionnalités
des produits Tobii.

\bigskip
Cette bibliothèque est donc idéale pour visualiser les données récupérées grace
àl'oculométrie, de façon simple et efficace. Cette libraire est assez bien
documentée aussi. PyGaze est donc vraiment intéressante à utiliser, en
combinaison avec les capteurs Tobii pour tirer profit d'un maximum de données
et créer des images de synthése les plus concluantes possibles. Ces-dernieres
sont très importantes pour permettre de tirer les bonnes conclusions.

\begin{figure}[htpb]
  \includegraphics[width=\textwidth,keepaspectratio=true]{heatmap.png}
  \caption{Exemple de heatmap sur une page de recherche sur Google}
\end{figure}

\subsection{GazeParser}



\begin{itemize}
  \item NVGaze
  \item https://imotions.com/blog/10-terms-metrics-eye-tracking/
  \item https://www.gazept.com/product/gazepoint-analysis-professional-edition-software/
  \item https://imotions.com/blog/free-eye-tracking-software/
\end{itemize}

% ----------------------------------------------------------------------------
% CONCLUSION
% ----------------------------------------------------------------------------

\section{Conclusion}

% ----------------------------------------------------------------------------
% BIBLIOGRAPHIE
% ----------------------------------------------------------------------------

\pagebreak
\bibliographystyle{unsrt}
\bibliography{recherches}

\end{document}
