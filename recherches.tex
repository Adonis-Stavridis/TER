\documentclass[12pt]{article}

\usepackage{lmodern}
\usepackage[utf8]{inputenc}
\usepackage[T1]{fontenc}
\usepackage[french]{babel}
\usepackage{url}
\usepackage{graphicx}
\usepackage{pifont}
\graphicspath{ {./img/} }

\newcommand{\cmark}{\ding{51}}
\newcommand{\xmark}{\ding{55}}

\title{\textbf{Analyse de données d’eye-tracking en Réalité Virtuelle}}
\author{\Large{Adonis Stavridis}}
\date{Février 2021}

\begin{document}

% ------------------------------------------------------------------------------
% TITLEPAGE
% ------------------------------------------------------------------------------

\maketitle
\tableofcontents
\pagebreak

% ------------------------------------------------------------------------------
% INTRODUCTION
% ------------------------------------------------------------------------------

\section{Introduction}

L'eye-tracking \cite{wiki:eye_tracking}, ou oculométrie, est une technologie
assez récente qui détermine la position du regard d'un individu dans un
environment virtuel. Elle établie alors une nouvelle interface entre Homme et
machine et c'est devenue aujourd'hui une technologie principale dans des études
liées au système visuel humain, à la psychologie, au marketing et au design.
Elle est aussi dèjà très utilisée dans les jeux vidéos. L'eye-tracking est très
ambitieux, cependant les recherches dans le domaine de la réalité virtuelle
reste encore moindres.

\bigskip
Déterminer la position du regard d'un individu dans un environment permet
d'effectuer des études quantitatives et qualitatives sur de multiples supports,
et ainsi comprendre les comportements humains dans différentes situations.
L'eye-tracking s'avère donc être très pratique pour étudier les zones qui sont
le plus attrayantes et celles qui le sont moins. Cependant, certaines études
ont besoin d'un environment plus réaliste. La réalité virtuelle ajoute une
nouvelle couche d'immersion, permettant à un individu de se sentir et agir de
façon plus réaliste. La réalité virtuelle permettrait alors de livrer des
résultats beaucoup plus fiables pour certains domaines, et ainsi pousser à
l'avancement des recherches sur le comportement humain.

\bigskip
En vue de ce travail de recherche, un premier environment a déjà été mis au
point pour recueillir des données d'eye-tracking sur un écran et en réalité
virtuelle. L'objectif est donc de réaliser la visualisation et l'analyse des
données oculaires, en temps réel ou par l'intermédiaire d'une application. Or,
il existe certaines bibliothèques qui permettent de visualiser ces données. Donc
laquelle est la plus adaptée pour traiter des données d'oculométrie sur un écran
et en réalité virtuelle?

\bigskip
Il faudrait d'abord comprendre comment sont récupérées les données, quelle est
leur forme et puis énumérer les type de visualisation des données qu'il est
souhaité de réaliser afin de déterminer la bibliothèque la plus intéressante à
utiliser pour accomplir ce travail d'analyse.

\pagebreak

% ------------------------------------------------------------------------------
% CAPTEURS
% ------------------------------------------------------------------------------

\section{Capteurs}

L'eye-tracking est permis grâce à des capteurs spéciaux qui envoient des rayons
infrarouges vers les yeux d'un individu. Ils sont infrarouges car ils ne sont
pas visibles par la vision humain. Ces rayons sont alors réfléchis par les yeux
et les capteurs peuvent ainsi déterminer la direction de regard au fil du temps.
Ensuite un logiciel intégré aux capteurs va déterminer la position du regard
dans un environment d'étude, que ce soit un écran ou un monde virtuel.

\bigskip
Il existe trois types de capteurs d'oculométrie sur la marché actuellement.
\begin{itemize}
  \item Des capteurs pour écrans : ce sont les capteurs les plus basiques. Ils
        sont placés au-dessus ou en-dessous d'un écran et orientés vers les yeux
        d'un individu. Ils permettent de suivre le regard sur un écran, donc
        dans un environment à deux dimensions.
  \item Des capteurs intégrés à des lunettes : ces capteurs suivent le regard
        d'un individu effectuant des tâches dans le monde réel. Les capteurs
        sont alors orientés vers les yeux de l'individu et des caméras
        enregistrent ce que l'individu observe devant lui.
  \item Des capteurs intégrés à des casques de réalité virtuelle : ce sont des
        capteurs similaires à ceux des écrans, mais qui ont été modifiés pour
        s'intégrer dans des casques de réalité virtuelle et observer les yeux à
        une distance beaucoup plus courte. Ces capteurs permettraient alors de
        suivre le regard dans un environment à trois dimensions. Un avantage à
        cette technologie dans la réalité virtuelle est le rendu fovéal
        \cite{wiki:foveated_rendering}. Les yeux ne se focalisent qu'en une
        seule région tandis que le reste du champ visuel est flou, donc cette
        méthode de rendu ne propose d'effectuer qu'un rendu de très haute
        qualité à la région de fixation mais une qualité de rendu moins
        importante dans les reste du champ périphérique.
\end{itemize}

\bigskip
La plupart des produits liés à l'eye-tracking sont distribués par des
entreprises telles que Tobii \cite{tobii} et HTC Vice \cite{htc_vive_pro_eye}.
Ces solutions peuvent être utilisées dans plusieurs domaines \cite{yt:tobii_vr}
: les capteurs pour écrans peuvent être utiles pour étudier le regard d'un
individu sur une page web, par exemple, et ainsi déterminer les régions les
plus attrayantes ou non. Les lunettes avec capteurs intégrés sont plutôt
utilisés dans le monde professionnel, dans les métiers de l'artisanal, dans la
médicine ou le pilotage. Les casques de réalité virtuelle avec capteurs
intégrés seraient utilisés dans des situations d'entraînement ou des
circonstances pas facilement réalisables dans le monde réel. Certains produits
similaires peuvent aussi être utilisés pour assister à des personnes en
situation de handicap et à mobilité réduite.

\bigskip
Il existe alors une grande variété des capteurs, certains plus spécialisés dans
des domaines que d'autres, mais ayants tous le même objectif. Ce suivi est
d'autant plus utile pour effectuer des recherches. Les capteurs sont fournis
avec des logiciels capables de produire toutes les données nécessaires
pour les analyser. Dans le cadre de ce travail, les capteurs utilisés sont une
barre de tracking Tobii pour écrans et un casque HTC Vive avec capteur intergré
pour la réalité virtuelle.

% ------------------------------------------------------------------------------
% ANALYSE DES DONNÉES
% ------------------------------------------------------------------------------

\section{Analyse des données}

Après avoir récupéré les données brutes d'une séance d'oculométrie, il serait
intéressant des les analyser afin d'en tirer des conclusions pertinantes. Les
données sont une série de nombres, représentants les positions du regard dans un
environment, mais sont difficilement compréhensibles. Pour cela, il serait
beaucoup plus intéressant de transformer les données en des réprésentations
graphiques de différents types. Il existe différentes mesures et termes
utilisés pour analyser le regard \cite{imotions:metrics}, chacune organisant
les données de façon à avoir une analyse compléte, étudiant tous les aspects du
regard.

\subsection{Points de fixation}

L'un des termes les plus utilisés pour représenter le regard au fil du temps
sont les points de fixation. Ce sont des mesures qui consistent à indiquer
le point de focalisation d'un individu à un instant donné. Un capteur qui
collectionne des données avec un taux d'échantillonage de 120hz, par exemple, va
fournir en sortie 120 points de fixations par seconde. Il faut donc faire une
organisantion des points en fonctions du temps et de l'espace. La quantité de
points de fixations montre à quel point l'attention visuelle a été portée en une
région.

\subsection{Séquence de fixations}

Étudier la séquence de ces points de fixations serait également intéressant. Un
individu va d'abord fixer son regard sur les régions les plus prioritaires et va
se construire l'environment petit à petit, en se concentrant sur différentes
zones. L'ordre d'attention est importante dans la recherche car elle reflète de
l'intérêt d'un individu sur une région et met en avant les objets les plus
captivants au premier regard. Le premier point de fixation peut être aléatoire,
mais les zones de fixations qui suivent peuvent parfois être prédites (pour
une interface utilisateur, par exemple). C'est donc un support qui permet d'en
tirer plusieurs conclusions.

\subsection{Heatmaps}

Une autre représentation très importante est la distribution des points de
fixation, sous la forme d'une heatmap. Un gradient de couleur est placée sur la
environment original indiquant du rouge au bleu les régions les plus
attrayantes (cf. Figure \ref{fig:heatmap}). Ces heatmaps sont un moyen très
simple de différencier les régions qui attirent le plus d'attention. C'est
l'une des représentations les plus importantes.

\begin{figure}[ht]
  \centering
  \includegraphics[width=0.75\textwidth,keepaspectratio=true]{heatmap.png}
  \caption{Exemple de heatmap sur une page de recherche sur Google
    \cite{img:heatmap}}
  \label{fig:heatmap}
\end{figure}

\subsection{Zones d'intérêt}

Une zone d'intérêt n'est pas vraiment une représentation mais un outil qui
permet de spécifier des zones, des objets au préalable, pour en extraire leurs
mesures. C'est un outil pratique pour bien séparer les différentes zones d'un
environment d'étude, et déterminer les régions les plus importantes. Dans les
mesures de chaque zone, il serait intéressant d'y calculer :

\begin{itemize}
  \item le délai à la première fixation : c'est le temps qu'un individu à mis
        depuis le début pour fixer une zone pour la première fois.
  \item la durée de fixation : c'est la somme du temps qu'un individu a passé a
        fixer une zone.
  \item le nombre de fixations : le regard peut fixer une zone, la quitter et
        puis revenir dessus; c'est donc le nombre de fois qu'un individu à fixé
        une zone.
\end{itemize}

\bigskip
Il existe donc plusieurs façons d'organiser les données et de les représenter
pour faciliter leur compréhension et permettre d'en tirer les bonnes
conclusions, en prenant en compte tous les aspects de l'étude du regard.

% ------------------------------------------------------------------------------
% LOGICIELS
% ------------------------------------------------------------------------------

\section{Logiciels}

Pour pouvoir analyser les données et créer ces réprésentations il faudrait soit
utiliser des bibliothèques ou des applications déjà disponibles
\cite{imotions:software}, ou sinon créer les outils. Heureusement, il existe
certaines des logiciels open-source, ou gratuits uniquement, permettant
d'effectuer un minimum d'analyse de données. L'objectif est donc de déterminer
quel logiciel est le plus intéressant à utiliser dans le cadre de ce travail,
pour analyser le regard sur un écran, mais surtout dans un environment de
réalité virtuelle.

\subsection{GazePointer}

Gazepointer \cite{gazepointer} est une application libre qui permet d'effectuer
de l'eye-tracking principalement. Il existe une fonctionnalité intégrée au
logiciel qui permet de faire des séances de suivi du regard, et génerer une
heatmap des fixations. Cependant c'est la seule analyse possible. De plus,
c'est un logiciel qui fonctionne indépendamment, et non pas une bibliothèque :
il ne permet pas à ses outils d'être accessibles ailleurs que dans
l'application. Aussi, le logiciel fonctionne uniquement avec des webcams.
Gazepointer permet d'analyser des données d'oculométrie, mais son
implémentation n'est pas optimale dans le cadre de ce travail, avec les
capteurs disponibles, et, en plus, ne permet pas du tout l'oculométrie en
réalité virtuelle.

\subsection{Ogama}

Ogama \cite{ogama} est une application open-source qui permet d'enregistrer et
d'analyser des données d'oculométrie et de mouvements de souris. Le logiciel
permet essentielment de créer des cartes de fixations et des zones d'intérêt,
et du calcul de saillance, pour déterminer quelles régions sont les plus
attrayantes. Il accepte toutes de données enregistrées en format ASCII, et, en
plus, le logiciel est certifié de fonctionner avec plusieurs logiciels
d'oculométrie et capteurs sur le marché tels que Tobii. Ogama est donc
certainement intéressant pour pouvoir importer et analyser des données
facilement. Cependant, ce n'est pas une bibliothèque et ne permet donc pas
autant de flexibilité.

\subsection{PyGaze}

PyGaze \cite{pygaze} est quant à elle une toolbox Python, permettant également
d'effectuer du suivi de regard et analyser les données. Elle est facile à
utiliser, il suffit d'avoir des connaissances de programmation en Python.
La fonctionnalité la plus intéressante de cette bibliothèque est l'outil
d'analyse des données. PyGaze permet alors d'afficher les données sous
forme de cartes de fixations, des cartes de séquence de fixations, des heatmaps,
mais aussi sous forme brute pour rendre visible toute l'espace que la
trajectoire a traversé. Le code de PyGaze est open-source et il est donc
possible de se baser sur cette bibliothèque pour développer d'autres outils. De
plus, elle supporte des SDK telles que celle de Tobii. Les façons de
représenter les données sont multiples, et il serait donc possible de
l'utiliser pour créer des outils d'analyse en réalité virtuelle.

\subsection{GazeParser}

GazeParser \cite{gazeparser} est une bibliothèque d'analyse des données du
regard qui permet essentielment de récupérer les données et les visualiser sous
forme de graphiques. Ceci peut être pratique pour avoir une approche plus brute
des données. GazeParser permet aussi d'appliquer des filtres sur les données
pour réduire le bruit et effectuer une analyse plus correcte. Mis à part ces
fonctionnalités, la bibliothèque ne propose rien de plus intéressant et donc son
utilité est moindre dans le cadre de ce travail. L'objectif de ce travail est
d'obtenir des représentations plus visuelles que graphiques, mais GazeParser
pourrait tout à fait être utilisée pour filtrer et améliorer le jeu de données,
afin d'avoir une analyse plus concluante.

\begin{center}
  \begin{tabular}{|c||c|c|c|}
    \hline
    \multicolumn{4}{|c|}{Comparaison des logiciels}                            \\
    \hline
    Logiciels   & Bibliothèque & Analyse               & Documentation         \\
    \hline
    GazePointer & \xmark       & \cmark                & \cmark                \\
    Ogama       & \xmark       & \cmark \cmark  \cmark & \cmark \cmark         \\
    PyGaze      & \cmark       & \cmark \cmark  \cmark & \cmark \cmark  \cmark \\
    GazeParser  & \cmark       & \cmark                & \cmark \cmark         \\
    \hline
  \end{tabular}
\end{center}

Il existe donc plusieurs applications et bibliothèques gratuites permettant
d'effectuer une analyse sur les jeux de données créés par les capteurs. Il
existe également des logiciels payants, mais ceux-ci sont hors de portée dans le
cadre de ce travail. L'objectif étant d'obtenir des représentations graphiques
des données, certaines bibliothèques sont plus intéressantes. PyGaze est le
candidat idéal ici, puisque c'est une bibliothèque et non une application, mais
aussi puisqu'elle propose beaucoup d'outils d'analyse des données et reste bien
documentée également.

% ----------------------------------------------------------------------------
% CONCLUSION
% ----------------------------------------------------------------------------

\pagebreak

\section{Conclusion}

L'eye-tracking est une technologie assez bien recherchée, mais les études dans
le domaine de la réalité virtuelle restent moindres. L'objectif de ce travail
étant de développer des outils d'analyse de données d'oculométrie, il faut
choisir un logiciel capable de créer des représentations pour des environments
à deux dimensions et de réalité virtuelle également. La bibliothèque la plus
intéressante à utiliser est PyGaze : elle nous permet de créer plusieurs cartes
d'analyse et pourrait être une très utile pour développer des outils d'analyse
en réalité virtuelle. Grâce aux capteurs Tobii et HTC Vive, il sera possible
d'enregistrer des données, qui seront analysés par la suite pour enfin créer des
représentations claires et précises de l'expérience, afin d'en tirer les bonnes
conclusions.

% ----------------------------------------------------------------------------
% BIBLIOGRAPHIE
% ----------------------------------------------------------------------------

\pagebreak
\bibliographystyle{unsrt}
\bibliography{recherches}

\end{document}
